\documentclass{letter}
\signature{Gregory John Matheson, \\ Contract Lecturer, \\ National United University}
\address{Language Center of National United University \\ 1, Lien Da, Kung-Ching Li, Miao-Li 36003 \\ Taiwan, ROC \\ (E-mail: drbean@freeshell.org) }

\begin{document}
\begin{letter}{
To Keng Language School,\\ 2F., No.231, ZiZhi Rd.,\\ Miaoli City\\ Taiwan, R.O.C}
\opening{Dear Sir/Madam,}

I found your school on http://bsb.edu.tw. Are you looking for a dedicated, experienced, effective native-speaker teacher?

For the last four years, I have been working at National United University. But next semester, NUU will be replacing me with an assistant professor. So I am looking for another job. I wish to apply for a position at your school. I think I would be an asset to your English program, teaching either kids or adults.

For the first ten years I was in Taiwan at Chinmin Institute of Technology, I was working with lower-level, unmotivated students. For these students, I developed a number of innovative teaching activities. One involved paired dictation, where partners read a dialog, but at the same time write it down. Another was a multiple choice quiz in the form of a basketball/baseball relay. Members of a team stationed around the room mark a ball with the answers to questions posted on the wall, before throwing it to the team member at the front of the room, who competes to throw the ball into the right basket placed on top of the blackboard before it fills up.

These dictation, relay race and other activities I have developed would work well with kids, I think.

At NUU, the students have been served better by the education system. They are more able learners. Still, they have been trained to be passive. Getting them to use English in small groups is a challenge. I have continued to use dictation, putting it on the web, at http://web.nuu.edu.tw/~greg/DictationExercises.html. The students listen to stories from http://www.storycorps.org, or other Internet sources, and re-create the text by filling in the blanks for homework. Using a question-answering web application I am now developing, they also write questions about the stories and choose answers, and the application tells them if their question is grammatical, and if the answer is correct.

As you can see, I have been developing my own curriculum. I don't use textbooks in class (although I am willing to use one if it is required.) Textbooks have acted as a starting point from which the curriculum has developed. For example, the topics and much of the material of my Business English classes comes from Longman's Market Leader.
 
The problems of most of my students in Taiwan has been a lack of confidence, stress and other forms of discomfort, and a lack of adventurousness. These were the result of unenjoyable experiences in school learning English, the lack of contact with users of the language, and disappointment with their experiments with the language. It would be my aim to give kids enjoyable experiences with English, to allow them to develop a personal relationship with a native speaker and give them wings (or encourage them to fly) with the language, by valuing risk-taking with it. 

Leaving NUU now, I am not retiring and I am not slowing down. With the next step in my career, I want to change the Taiwan English-language situation, developing my teaching repertoire and helping my students grow as learners of English. I am confident the future is bright. If your school is also of such a view, I would enjoy working with you, developing confident, relaxed, adventurous and superior speakers of the English language.

\closing{Yours sincerely,}


\end{letter}
\end{document}
