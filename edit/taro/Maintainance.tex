Maintenance

Thanks to its simplicity,the shakuhachi is a very durable instrument, which will last more than one thousand years, if it is stored properly. 

1  Avoid sun light, and keep humidity at 55~75 per cent, to prevent cracking.
   Wiping the instrument with walnut/olive oil once every 3~4 months is helpful.
   Storing the shakuhachi in the bathroom is recommended, especially in a place like a hotel, or skyscraper, or wherever the air is deadly dry. 

2  After practice, dry the inside of the instrument with a cotton cloth.
               
   (photo of tsuyukiri) 

3  A crack should be repaired as soon as possible, as it will grow longer and deeper.
   When it gets to the inside of the instrument, the repair work costs a lot.

4  If the lacquer peels off, it should be japanned, before you play the 
   instrument again. This problem causes the worst splitting of the shakuhachi.


                                  end




	
