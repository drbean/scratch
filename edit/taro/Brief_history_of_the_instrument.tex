
Brief history of the instrument

The Shakuhachi is a vertical flute made of bamboo. It was introduced from China around the 12th century. At that period, the instrument was used to play the Chinese court music called Gagaku. In medieval times, the shakuhachi became popular 
among zen monks and samurai. Many conversions were made to the instrument at this time, to make the music suitable
for Japanese sensitivity. Especially, a group of zen monks called komuso developed their compositions into a form of    
meditation called suizen (blowing zen). All these classic pieces remain today, showing the peak of the spirituality and the 
technique of shakuhachi music. After the abolition of komuso temples (1871), the shakuhachi was released from the monopoly of the komuso, and started to be used in folk music, in concert with koto and shamisen, and even in western 
music. One distinctive feature of this simple instrument is that it obtained the ability to produce all the notes
of the keyboard, already in medieval times. The basic structure has not been changed for 7~800 years, as if it is at the end of its evolution.

                                                 end 
