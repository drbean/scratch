\documentclass[a4paper]{article}
\usepackage[T1]{fontenc}
% \usepackage[absolute,noshowtext,showboxes]{textpos}
\usepackage[absolute,showboxes]{textpos}
% \textblockorigin{0.10cm}{1.00cm}
\textblockorigin{0.00cm}{0.00cm} %HPLaserJet5000LE
\usepackage{texdraw}
\pagestyle{empty}
\setlength{\unitlength}{1cm}

\newcommand{\myIdentifier}[0]{
graham 11
}

\newcommand{\myAcontent}[0]{
It's important to have clear goals, but it's also important to ask users what they think. Planning and learning from users goes together. It is very difficult to plan, because you can't see the future. But if you have thought very hard about WHAT you want to do and have a very clear goal, it is easier to decide HOW to achieve your goal.

}

\newcommand{\myBcontent}[0]{
Writing down business plans is hard work, but it will help you be clear about what your business is doing. The written plan can also be shown to people who want to learn about your business. The hard work writing down the business plan is not a waste of time. It helps you decide what your goals are.

}

\newcommand{\myCcontent}[0]{
In the best businesses, all the people in the business are excited about the business's plans. But how does the boss get everyone to feel excited about his/her plans? He/she has to talk with everyone about the business's goals, and everyone has to feel the boss is listening to what they think. The boss has to listen to everyone INSIDE the company as well as to users OUTSIDE the company.

}

\newcommand{\myDcontent}[0]{
William Kendall became very good at planning ahead. He couldn't sell his previous company, because people thought they wouldn't run it as well as him. So, starting his next company, he immediately started thinking how to sell it. He went to a very big company, and asked them if they wanted to invest ten percent. They did. And now the big company has bought all of his company. That was his plan.

}

\newcommand{\mycard}[5]{%
	\vspace{0.1cm}
	\small #1 #2
	\par
	\parbox[t][6.7cm][c]{9.5cm}{%
	\hspace{0.1cm} \Large#3\\
	\normalsize#4 #5
	}
}

\begin{document}
\fontfamily{hlst}\fontseries{b}\fontshape{n}\selectfont
% \begin{picture}(15,20)(+4.8,-22.05)
% \begin{tabular}[t]{*{2}{|p{10.05cm}}|}

\begin{textblock}{8}(0,0)
\textblocklabel{picture1}
\mycard{}{\myIdentifier}{\parbox{9.0cm}{A:
\myAcontent
}}{}{} 
\end{textblock}

\begin{textblock}{8}(8,0)
\textblocklabel{picture2}
\mycard{}{\myIdentifier}{\parbox{9.0cm}{B:
\myBcontent
}}{}{} 
\end{textblock}

\begin{textblock}{8}(0,4)
\textblocklabel{picture3}
\mycard{}{\myIdentifier}{\parbox{9.0cm}{C:
\myCcontent
}}{}{} 
\end{textblock}

\begin{textblock}{8}(8,4)
\textblocklabel{picture4}
\mycard{}{\myIdentifier}{\parbox{9.0cm}{D:
\myDcontent
}}{}{} 
\end{textblock}

\begin{textblock}{8}(0,8)
\textblocklabel{picture5}
\mycard{}{\myIdentifier}{\parbox{9.0cm}{A:
\myAcontent
}}{}{} 
\end{textblock}

\begin{textblock}{8}(8,8)
\textblocklabel{picture6}
\mycard{}{\myIdentifier}{\parbox{9.0cm}{B:
\myBcontent
}}{}{} 
\end{textblock}

\begin{textblock}{8}(0,12)
\textblocklabel{picture7}
\mycard{}{\myIdentifier}{\parbox{9.0cm}{C:
\myCcontent
}}{}{} 
\end{textblock}

\begin{textblock}{8}(8,12)
\textblocklabel{picture8}
\mycard{}{\myIdentifier}{\parbox{9.0cm}{D:
\myDcontent
}}{}{} 
\end{textblock}

\end{document}


