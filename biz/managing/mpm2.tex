\documentclass[a4paper]{article}
\usepackage[T1]{fontenc}
% \usepackage[absolute,noshowtext,showboxes]{textpos}
\usepackage[absolute,showboxes]{textpos}
% \textblockorigin{0.10cm}{1.00cm}
\textblockorigin{0.00cm}{0.10cm}
% \textblockorigin{0.00cm}{0.00cm} %HPLaserJet5000LE
\usepackage{texdraw}
\pagestyle{empty}
\setlength{\unitlength}{1cm}

\newcommand{\myIdentifier}[0]{
MPM 3
}

\newcommand{\myAcontent}[0]{
The Sales Manager thinks the Muller sales reps are better sales people because their main aim is to sell as much as possible. The manager likes that they are very competitive and they put a lot of pressure on customers. He likes that they are happy with the low basic salary and high commissions. He doesn't like the Peterson sales team because it doesn't do this.
}

\newcommand{\myBcontent}[0]{
The Assistant Sales Manager likes the Peterson sales reps better because they build better customer relationships. Customers trust them. The sales reps do not pressure them and they share information about them with each other by writing good reports. He doesn't like the Muller sales team because it doesn't share information and its sales reports are not good.
}

\newcommand{\myCcontent}[0]{
Bad things about the Muller sales team are its promises of early delivery dates, even if the company cannot meet the date. This is lying to customers and they complain. It also doesn't share information with other reps. Its reports are useless to them. It gives individual buyers expensive gifts and so relationships with customers are not open and fair.
}

\newcommand{\myDcontent}[0]{
Bad things about the Peterson sales team: It's not enthusiastic about making sales. It wants higher salaries and bonuses if sales targets are met. It's not competitive. It does not promise early delivery dates. It does not put any pressure on customers at all. It does not make as much money for the company as the Muller sales team.
}

\newcommand{\mycard}[5]{%
	\vspace{0.1cm}
	\small #1 #2
	\par
	\parbox[t][6.7cm][c]{9.5cm}{%
	\hspace{0.1cm} \Large#3\\
	\normalsize#4 #5
	}
}

\begin{document}
\fontfamily{hlst}\fontseries{b}\fontshape{n}\selectfont
% \begin{picture}(15,20)(+4.8,-22.05)
% \begin{tabular}[t]{*{2}{|p{10.05cm}}|}

\begin{textblock}{8}(0,0)
\textblocklabel{picture1}
\mycard{}{\myIdentifier}{\parbox{9.0cm}{A:
\myAcontent
}}{}{} 
\end{textblock}

\begin{textblock}{8}(8,0)
\textblocklabel{picture2}
\mycard{}{\myIdentifier}{\parbox{9.0cm}{B:
\myBcontent
}}{}{} 
\end{textblock}

\begin{textblock}{8}(0,4)
\textblocklabel{picture3}
\mycard{}{\myIdentifier}{\parbox{9.0cm}{C:
\myCcontent
}}{}{} 
\end{textblock}

\begin{textblock}{8}(8,4)
\textblocklabel{picture4}
\mycard{}{\myIdentifier}{\parbox{9.0cm}{D:
\myDcontent
}}{}{} 
\end{textblock}

\begin{textblock}{8}(0,8)
\textblocklabel{picture5}
\mycard{}{\myIdentifier}{\parbox{9.0cm}{A:
\myAcontent
}}{}{} 
\end{textblock}

\begin{textblock}{8}(8,8)
\textblocklabel{picture6}
\mycard{}{\myIdentifier}{\parbox{9.0cm}{B:
\myBcontent
}}{}{} 
\end{textblock}

\begin{textblock}{8}(0,12)
\textblocklabel{picture7}
\mycard{}{\myIdentifier}{\parbox{9.0cm}{C:
\myCcontent
}}{}{} 
\end{textblock}

\begin{textblock}{8}(8,12)
\textblocklabel{picture8}
\mycard{}{\myIdentifier}{\parbox{9.0cm}{D:
\myDcontent
}}{}{} 
\end{textblock}

\end{document}


