\documentclass[a4paper]{article}
\usepackage[T1]{fontenc}
% \usepackage[absolute,noshowtext,showboxes]{textpos}
\usepackage[absolute,showboxes]{textpos}
% \textblockorigin{0.10cm}{1.00cm}
\textblockorigin{0.00cm}{0.00cm} %HPLaserJet5000LE
\usepackage{texdraw}
\pagestyle{empty}
\setlength{\unitlength}{1cm}

\newcommand{\myIdentifier}[0]{
German-American smiling
}

\newcommand{\myAcontent}[0]{
You are an American tourist buying a beer stein as a souvenir at a German store. The woman who sells you the glass is unfriendly and unpleasant. She doesn't smile at you. Thinks: "She doesn't like Americans? She doesn't think the stein is a good souvenir? If she is in a bad mood, she shouldn't make other people feel bad too!" None of the German service people you've met have been welcoming.
}

\newcommand{\myBcontent}[0]{
You are sales clerk in a German store that sells kitchenware and diningware. You have never had any foreigners in your store. An American tourist comes in. She smiles at you. She doesn't know you. Thinks: "I don't like this woman. Is she going to ask me if she can use the bathroom? Does she want me to be her friend? Did I do something embarrassing? She is insincere, maybe even dishonest."
}

\newcommand{\myCcontent}[0]{
1. A and B don't show any interest in the culture of the other person. They are only thinking of the effect on themselves. They are ethnocentric. Bad: They don't relativize their own values. 2. A knows a little of B's customs. B doesn't know about A's customs. 4. They don't try to talk to each other and understand the rules for smiling. Bad: A knows German service people don't smile, but is still upset by B.
}

\newcommand{\myDcontent}[0]{
3. Good: They try to relate the other person's behavior to behavior they know in their own culture. Bad: They are not able to see the other person's behavior as related to their own. 5. Bad: They don't try to question their own beliefs about smiling and how they affect their reactions. Bad: They question each other's behavior but without a basis for criticism.
}

\newcommand{\mycard}[5]{%
	\vspace{0.1cm}
	\small #1 #2
	\par
	\parbox[t][6.7cm][c]{9.5cm}{%
	\hspace{0.1cm} \Large#3\\
	\normalsize#4 #5
	}
}

\begin{document}
\fontfamily{hlst}\fontseries{b}\fontshape{n}\selectfont
% \begin{picture}(15,20)(+4.8,-22.05)
% \begin{tabular}[t]{*{2}{|p{10.05cm}}|}

\begin{textblock}{8}(0,0)
\textblocklabel{picture1}
\mycard{}{\myIdentifier}{\parbox{9.0cm}{A:
\myAcontent
}}{}{} 
\end{textblock}

\begin{textblock}{8}(8,0)
\textblocklabel{picture2}
\mycard{}{\myIdentifier}{\parbox{9.0cm}{B:
\myBcontent
}}{}{} 
\end{textblock}

\begin{textblock}{8}(0,4)
\textblocklabel{picture3}
\mycard{}{\myIdentifier}{\parbox{9.0cm}{C:
\myCcontent
}}{}{} 
\end{textblock}

\begin{textblock}{8}(8,4)
\textblocklabel{picture4}
\mycard{}{\myIdentifier}{\parbox{9.0cm}{D:
\myDcontent
}}{}{} 
\end{textblock}

\begin{textblock}{8}(0,8)
\textblocklabel{picture5}
\mycard{}{\myIdentifier}{\parbox{9.0cm}{A:
\myAcontent
}}{}{} 
\end{textblock}

\begin{textblock}{8}(8,8)
\textblocklabel{picture6}
\mycard{}{\myIdentifier}{\parbox{9.0cm}{B:
\myBcontent
}}{}{} 
\end{textblock}

\begin{textblock}{8}(0,12)
\textblocklabel{picture7}
\mycard{}{\myIdentifier}{\parbox{9.0cm}{C:
\myCcontent
}}{}{} 
\end{textblock}

\begin{textblock}{8}(8,12)
\textblocklabel{picture8}
\mycard{}{\myIdentifier}{\parbox{9.0cm}{D:
\myDcontent
}}{}{} 
\end{textblock}

\end{document}


