\documentclass[a4paper]{article}
\usepackage[T1]{fontenc}
% \usepackage[absolute,noshowtext,showboxes]{textpos}
\usepackage[absolute,showboxes]{textpos}
% \textblockorigin{0.10cm}{1.00cm}
\textblockorigin{0.00cm}{0.00cm} %HPLaserJet5000LE
\usepackage{texdraw}
\pagestyle{empty}
\setlength{\unitlength}{1cm}

\newcommand{\myIdentifier}[0]{
online retailing
}

\newcommand{\myAcontent}[0]{
The Internet has had a big effect on retailing. Online prices are very cheap because retailers do not have to run stores. Websites that publish price comparisons make the market even more competitive.
Some retailers tried charging different prices for goods on their websites and in their stores. But this is now hard to do. People want retailers to integrate their operations so they can buy online and pick up their purchases in stores.

}

\newcommand{\myBcontent}[0]{
To be successful, the website must be easy to find and look good. It must have good images and lots of information about the things on sale. There needs to be clear information about prices and special offers. Information about delivery of the products is very important. Also the site must be fast and easy to navigate. It must be easy to find things on the site and it must be easy to order the goods and pay for them.
}

\newcommand{\myCcontent}[0]{
Not everyone likes buying online. Some want to see the real thing, not a picture. Some think their money will be stolen. Other people like going shopping.
Customers who also like to use websites are also different. And even the same customer is different on different days. Sometimes they will look at the website and then go into a store. Sometimes they will order online. 

}

\newcommand{\myDcontent}[0]{
Argos is a big retailer with many stores all over the U.K. which also has an online operation. It tries to integrate its two different forms of retailing, because its experience is that sometimes a customer will buy online and at other times the same customer will check the website to find information and then go into a store to buy the item.
}

\newcommand{\mycard}[5]{%
	\vspace{0.1cm}
	\small #1 #2
	\par
	\parbox[t][6.7cm][c]{9.5cm}{%
	\hspace{0.1cm} \Large#3\\
	\normalsize#4 #5
	}
}

\begin{document}
\fontfamily{hlst}\fontseries{b}\fontshape{n}\selectfont
% \begin{picture}(15,20)(+4.8,-22.05)
% \begin{tabular}[t]{*{2}{|p{10.05cm}}|}

\begin{textblock}{8}(0,0)
\textblocklabel{picture1}
\mycard{}{\myIdentifier}{\parbox{9.0cm}{A:
\myAcontent
}}{}{} 
\end{textblock}

\begin{textblock}{8}(8,0)
\textblocklabel{picture2}
\mycard{}{\myIdentifier}{\parbox{9.0cm}{B:
\myBcontent
}}{}{} 
\end{textblock}

\begin{textblock}{8}(0,4)
\textblocklabel{picture3}
\mycard{}{\myIdentifier}{\parbox{9.0cm}{C:
\myCcontent
}}{}{} 
\end{textblock}

\begin{textblock}{8}(8,4)
\textblocklabel{picture4}
\mycard{}{\myIdentifier}{\parbox{9.0cm}{D:
\myDcontent
}}{}{} 
\end{textblock}

\begin{textblock}{8}(0,8)
\textblocklabel{picture5}
\mycard{}{\myIdentifier}{\parbox{9.0cm}{A:
\myAcontent
}}{}{} 
\end{textblock}

\begin{textblock}{8}(8,8)
\textblocklabel{picture6}
\mycard{}{\myIdentifier}{\parbox{9.0cm}{B:
\myBcontent
}}{}{} 
\end{textblock}

\begin{textblock}{8}(0,12)
\textblocklabel{picture7}
\mycard{}{\myIdentifier}{\parbox{9.0cm}{C:
\myCcontent
}}{}{} 
\end{textblock}

\begin{textblock}{8}(8,12)
\textblocklabel{picture8}
\mycard{}{\myIdentifier}{\parbox{9.0cm}{D:
\myDcontent
}}{}{} 
\end{textblock}

\end{document}


